\documentclass[10pt]{IEEEtran} % .... using IEEE Transaction style
\usepackage{graphics}
\usepackage[cmex10]{amsmath}

% the following is to get inch margins on a letter-size paper
\setlength{\topmargin}{0pt}
\setlength{\headheight}{0in}
\setlength{\headsep}{0in}
\setlength{\textheight}{9.0in}
\setlength{\footskip}{0.5in}
\setlength{\oddsidemargin}{0pt}
\setlength{\evensidemargin}{0pt}
\setlength{\textwidth}{6.5in}

\begin{document}
\title{Scheduling Simultanious Events for Multple Mobile Devices}
\author{Jacob Schwartz}
\maketitle

\begin{abstract}
*HTTP adaptive streaming protocols like HTTP Live Streaming (HLS) and Dynamic 
Adaptive Streaming over HTTP (DASH) are gaining traction amongst mobile clients.
This protocols can act agressively on a client, wanting to grab the best
possible video quality possible. This can even be at the expense of other
clients on the network. In order to enforce class of service (CoS), algorithms
try to tame the client's aggressiveness by making them downshift video bitrates
under poor network conditions. In this paper, an alternative downshifting
formula is discussed, along with a small experiment that looks at the download
times for video segment files at different bitrates under a real-world WiFi
network.


In this paper, a basic network time synchronization technique will be used to
schedule two mobile devices to take a picture at the same time.
\end{abstract}

\section{Introduction}
*Using HTTP adaptive streaming protocols, mobile devices are able to quickly
adapt to the changing networks they travel through. This creates a problem as
the mobile client will aggressively attempt to take the available bandwidth to 
download the best available bandwidth. If there is no one on the network, this
is fine, but the clients also can act aggressively when the network is congested
with traffic, possibly other aggressive clients. When they are overly 
aggressive, the clients break the class of service (CoS) infrustructure. When 
CoS is not upheld, all clients on a network suffer. 

*To counter this problem, additional logic must be implemented into the client.
This logic examines the current state of the network and adjusts the client's
aggressiveness accordingly. When the network is not being fully utilized, the
client acts as agressively as possible by retriving the highest quality video
segments. Inversly, if the network is being utilized, the client takes notice
and drops the bitrate one level. Segment downloads also are aborted if it is
taking too long to recieve the segment. The bitrate will also be dropped in this
case and will be restored if the network recovers.

In this paper, we use a basic process to calculate the time offsets between two
mobile devices. We schedule an event on one of the devices and it tells the
other device to schedule the same event for the same time, using the time offset
of the device. To show the these events occur at the same point in time, both
device take a picture of the same clock. To take the experiment a step further,
the scheduling of events could also take place across different platforms or run
off different wireless protocols like TCP or Bluetooth.

\section{Prior Work}

*To provide the best user experience when watching streaming video using one of
the HTTP adaptive streaming protocols, rate adaption is used. This gives the
user the highest quality video possible while still maintains constant buffering
on the client. Lui et al. \cite{rate} approach this problem at the video segment level. 
They pace the client requests on a timer to keep the network from getting
congested with several requests coming all at once. In addition, they also 
propose a TCP-level solution but most mobile clients do not operate a TCP buffer
level, but instead work at the segment level.

*Controlling the timing of the requests does not alter the agressiveness of the
client as it still may be attempting to download a bitrate that is still too
high for the current network. Ma and Bartos \cite{cos} add request backoffs and download 
aborts for segments that are late so that they can correctly select the 
agressiveness of the client. This allows for better class of service (CoS) 
enforcement. In their case, the agressiveness of the client directly translates 
into the bitrate ($b$) of video it tries to download. If there is extra 
bandwidth on the network, the client will try to to upshift to a better bitrate. 
Likewise, when the client is using too much bandwith or their is a lot of 
traffic on the network and the segment does not download in time, the bitrate is 
downshifted.

*An upshift ($u$) is determined by the possibility of moving up to the next bitrate
based of the most recent segments download time $\delta$. Whereas downshifts 
($d$) only occur when a download abort for segments that are taking too long. 
The segment abort time $\alpha$ is typically set to $L$, which is the length of 
all of the segments in seconds.
\begin{equation}
    d = \delta > \alpha
\end {equation}
\begin{equation}
    u = \frac{b+1}{b} \times \delta > \alpha
\end {equation}

\section{Improved Downshifting}

*Having to wait until a segment download is aborted before making a bandwidth
downshift is a waste of time. A decision could be made before the next segment 
on whether or not to downshift. Knowing the time taken to download the previous
segment and the network conditions at the time, a guess can be made about the 
download time of the next segment based on the current network conditions. 
In this experiment, I will be ignoring segment buffering and queue behaviors 
discussed by Ma and Bartos.
\begin {equation}
    d = \delta_t > \alpha \vee E_{t - 1} \times \delta_{t - 1} < E_{t} \times L
\end {equation}

*In the updated formula, a downshift will still occur if there is a download
abort due to a timeout. There is another condition that can cause a timeout that
is checked before the segment download even starts. The current excess bandwidth
on the network is multiplied by the the maximum time before a download abort
happens and is compared to the download time of the previous download conditions
and segment download time. This allows for a ratio of excess bandwidth to 
download time and when it reaches a certain threashold, a downshift will occur.

\section{Results}

To test the validity of our algorithm, we wrote a simple application to find 
the offset between two mobile devices, iOS devices in this case. This
application has one device declare itself the server. This device then waits for
a client device to connect. Over a UDP connection, the server then sends the
device's time in milliseconds since the Unix Epoch several times, which the
client uses to get an average offset from its time. Once this offset is
calculated, the user tells the client device when to take a picture. The client
tells the server to take a picture at the same point in time, using the average
offset that was calculated. To determine how far off the picture event take
place, the phones are positioned to take a picture of a digital clock.

%SCREEN SHOTS OF PICTURES AND SHIT

\begin{figure}
{\resizebox{3.2in}{!}{\includegraphics{experiment/out-0.png}}}
\caption{Download times at 0 Kbps}
\label{fig:graph0}
\end{figure}

\section{Conclusion}

*These times were obtained on a laptop using a wireless network with little to 
no other traffic at the same time. Based on the times, it is clear that in this
setting there would be no need for downshifting. As seen in the Figures 1-5,
most of the times occur in under a second. Interestingly, the times for all of 
the Akumai segments with bandwidths that actually contain video download faster 
than the Azuki segments. This could happen for a number of reasons, closer 
physical location being one of them, but this is not the focus of the 
experiment.

There are several expansions to this project. One next step is to port the 
application to other mobile operating systems to find any other potential 
pitfalls that would cause the offset and the timing to be incorrect. Connecting
over Bluetooth would eliminate the 'ugliness' of having to enter an IP address
to connect. To use more than two devices, a multicast protocol could be used and
one of the mobile devices would act as a server or an actual dedicated server
could be used. The latter would also remove the need to enter an IP address,
because it would be static. 

\begin{thebibliography}{1}

\bibitem{rate}
C. Lui, I. Bouazizi, and M. Gabbouj, ``\emph{Rate Adaptation for Adaptive HTTP
Streamimg},'' ACM MMSys'11, February 2011.

\bibitem{cos}
K. Ma, and R. Bartos, ``\emph{CoS Enforcement for HTTP Adaptive Streaming},'' 
IEEE GC'12 Workshop, July 2003.

\end{thebibliography}

\end{document}
